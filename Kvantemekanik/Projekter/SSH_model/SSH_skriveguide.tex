\documentclass[10pt]{article}
\usepackage[T1]{fontenc}
\usepackage[english, danish]{babel}
\usepackage[utf8]{inputenc}
\usepackage{fullpage}
\usepackage{url}
\usepackage{varioref}
\usepackage{braket}
\usepackage{graphicx}
\usepackage{multicol}
\usepackage{amsmath,amssymb}
\usepackage{mathtools}	

\title{Skriveguide: Topologiske isolatorer, SSH-modellen.}
\author{Kristian Knakkergaard Nielsen}
\begin{document}
\maketitle
Målet med dette dokument er at give et forslag til, hvordan en rapport i projektet omkring topologiske isolatorer kan opbygges. Lignings- og figurreferencer refererer til SSH\_opgaver.pdf.

\section{Introduktion/Indledning}
Skal indeholde:
\begin{itemize}
\item Motivér hvorfor vi kigger SSH-modellen. 
\item Introducér krystalstruktur med udgangspunkt i SSH-modellen. Brug figur 2. 
\item Forklar kort de to analysemetoder: 1) analyse af `det indre' af materialet (bulk), og 2) analyse af kanten. 
\end{itemize}

Længde: 3/4 - 1 side. 

\section{Egentilstande og energibånd}
Skal indeholde:
\begin{itemize}
\item Introducér Hamiltonoperatoren i stedrum. Forklar kort Dirac-notationen og fysikken bag leddene. 
\item Forklar kort transformationen til impulsrum (Fourier-transformationen), herunder hvad $k$ er.
\item Opskriv Hamiltonoperatoren i impulsrum i matrixform (ligning (25)). 
\item Definér og opskriv $\mathcal{H}_k$ som $2\times 2$ matrix. Omskriv i det samme $\mathcal{H}_k$ vha. Paulimatricer og vektoren $\mathbf{h}_k$. 
\item Definér fasen $\phi_k$, og lav en tegning af denne ala figur 3. 
\item Opskriv egentilstandene i ligning (16). Forklar kort, at $H\ket{\psi_{\pm, k}} = \pm |\mathbf{h}_k|\ket{\psi_{\pm, k}}$. 
\item Plot energibåndene $E_{\pm, k} = \pm |\mathbf{h}_k|$ og forklar hvorledes disse fyldes efterhånden som man putter partikler ind i systemet. Understreg generalitet for faste stoffer.
\item Indtegn båndgabet, $\Delta E$, på plottet af energibåndene.  
\item Forklar, at båndgabet lukker, når $\delta t = 0$. Lav eventuelt en skitse heraf.
\item Beskriv herudfra, hvad man forstår ved en topologisk faseovergang. 
\end{itemize}

Kan indeholde:
\begin{itemize}
\item Beskrivelse af Blochs sætning i relation til ligning (18) umiddelbart efter, at egentilstandene er opskrevet. 
\end{itemize}

Længde: 2 sider. 

\section{Symmetri og topologi}
Skal indeholde:
\begin{itemize}
\item Introducér symmetrien, $S$. 
\item Opskriv på matrixform, ligning (25). Vis, at $SH = -HS$. Relatér dette til energierne og de fundne egentilstande. Henvis gerne til plottet af energibåndene her! 
\item Beskriv, hvilken restriktion det sætter på $\mathcal{H}_k$, hvis vi \textit{kræver}, at systemet overholder symmetrien $S$. 
\item Definér nu omdrejningstallet, $\mathcal{W}$. Forklar hvornår denne er en topologisk invariant. Henvis meget gerne til figuren, hvor I definerer fasen $\phi_k$. 
\item Beregn omdrejningstallet som funktion af $\delta t$ og bemærk at den skifter diskontinuert i $\delta t = 0$. Relatér til båndgabet. 
\end{itemize}

Længde: 1 1/2 sider. 


\section{Kanttilstanden}
Skal indeholde:
\begin{itemize}
\item Opskriv og forklar ligning (26), Hamiltonoperatoren for den åbne kæde. 
\item Bestem kanttilstanden for $\delta t > 0$. Understreg, at man ikke kan finde en for $\delta t < 0$. 
\item Skitsér tilstanden og forklar hvad der sker i grænserne $\delta t \to 0$ og $\delta t \to t$. 
\item Relatér til omdrejningstallet, $\mathcal{W}$. 
\end{itemize}

Længde: 1 side. 


\section{Konklusion}
Skal indeholde:
\begin{itemize}
\item En kort opridsning af, hvad vi har vist. 
\item Beskriv, at man ikke kan vide om man er i topologisk fase ud fra energibåndene. 
\item Beskriv, at man i stedet er nødt til at kigge på egentilstandene, og specifikt den topologiske invariant. 
\end{itemize}

Længde: højst 1/2 side. 

\end{document}