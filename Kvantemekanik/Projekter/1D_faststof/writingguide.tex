\documentclass[10pt]{article}
\usepackage[T1]{fontenc}
\usepackage[english, danish]{babel}
\usepackage[utf8]{inputenc}
\usepackage{fullpage}
\usepackage{url}
\usepackage{varioref}
\usepackage{braket}
\usepackage{graphicx}
\usepackage{multicol}
\usepackage{amsmath,amssymb}
\usepackage{mathtools}	

\title{Skriveguide, fononer}
\author{Kristian Knakkergaard Nielsen}
\begin{document}
\maketitle
Målet med dette dokument er at give et forslag til, hvordan en rapport i projektet omkring fononer kan opbygges. Ligningsreferencer refererer til linearchain\_quantization.pdf.

\section{Introduktion/Indledning}
Skal indeholde:
\begin{itemize}
\item Motivér hvorfor vi kigger på det system, vi gør. 
\item Forklar de relevante fysiske antagelser. (Born-Oppenheimer, harmonisk og nærmeste nabo approksimationerne)
\end{itemize}

Længde: 1/2 - 3/4 side. 

\section{Hamiltonoperatoren i stedrum og impulsrum}
Skal indeholde:
\begin{itemize}
\item Introducér Hamiltonoperatoren i stedrum. Forklar fysikken bag leddene.
\item Forklar kort transformationen til impulsrum. 
\item Opskriv Hamiltonoperatoren in impulsrum.
\item Understreg dispersionsrelationen $\omega(k)$. 
\item Med udgangspunkt i den harmoniske oscillator fra Griffiths: forklar $a$-operatorerne. Opskriv den endelige Hamiltonoperator.
\item Forklar, hvad en fonon er, og herunder hvad grundtilstanden er.
\item Forklar formen af $\omega(k)$ for lave $k$ (lydhastighed).
\end{itemize}

Kan indeholde:
\begin{itemize}
\item Udregning af grundtilstandsenergien. Relatér til uafhængige partikler. 
\item Lav en tegning af en harmoniske oscillator for at beskrive fononer nærmere.
\end{itemize}

Længde: 1 1/2 - 2 sider. 

\section{Tidsudvikling og oscillation}
Skal indeholde:
\begin{itemize}
\item Forklar Heisenberg-billedet kort.
\item Udled ligning (21). Forklar kort. 
\item Forklar ligning (22).
\item Forklar kort hvordan ligning (24) udledes. Beskriv fysikken bag denne ligning. 
\item Forklar kort hvordan ligning (25) udledes. 
\item Brug ligning (24) og (25) til lave en figur i stil med figur 1. Forklar hvad den viser.
\end{itemize}

Kan indeholde:
\begin{itemize}
\item En forklaring af ligning (21) som en feltekspansion. 
\end{itemize}

Længde: 1 1/2 - 2 sider. 


\section{Kontrolleret forstyrrelse eller Termodynamiske egenskaber}
Dette afsnit kan enten handle om kontrolleret forstyrrelse eller de termodynamiske egenskaber.

\subsection{Termodynamiske egenskaber}
Hvis dette afsnit vælges foreslår jeg at holde længden af afsnittene omkring Hamiltonoperatoren og tidsudviklingen på et minimum. 

Skal indeholde:
\begin{itemize}
\item Forklar partitionsfunktionen. Udled $Z_k$ (kort).
\item Udled ligning (34) og heraf ligning (35).
\item Beskriv tilstandstætheden (density of states) for systemet. 
\item Beregn analytisk den totale energi i de to grænser (høj og lav temperatur) og kommentér på højtemperaturgrænsen ud fra ligefordelingsloven.
\item Find herfra varmekapaciteten og relatér lavtemperaturgrænsen til termodynamikkens tredje lov.  
\end{itemize}

Kan indeholde:
\begin{itemize}
\item En grafisk sammenligning af de analytisk fundne grænser for den totale energi sammen med en numerisk beregning heraf. 
\end{itemize}

Længde: 1 - 2 sider.

\subsection{Kontrolleret forstyrrelse}
Skal indeholde:
\begin{itemize}
\item Opskriv ligning (27)
\item Forklar hvad $\ket{\alpha}$-tilstanden er. 
\item Forklar, hvordan ligning (28) udledes. 
\item Et par konkrete eksempler for $b_l$. F.eks.: $b_l = (-1)^l b$.
\end{itemize}
Længde: 3/4 - 1 side.


\section{Konklusion}
Skal indeholde:
\begin{itemize}
\item En kort opridsning af, hvad vi har vist. 
\item Herunder: hvad er den overordnede opførsel af $\omega(k)$ og den fysiske betydning heraf.
\item Hvordan kommer dispersionsrelationen til udtryk i de grafer, I har lavet? 
\item For kontrolleret forstyrrelse eller termodynamiske egenskaber: oprids kort resultaterne.
\end{itemize}

Længde: højst 1/2 side. 

\end{document}