\documentclass[10pt]{article}
\usepackage[T1]{fontenc}
\usepackage[english, danish]{babel}
\usepackage[utf8]{inputenc}
\usepackage{fullpage}
\usepackage{url}
\usepackage{varioref}
\usepackage{graphicx}
\usepackage{multicol}
\usepackage{amsmath,amssymb}
\usepackage{mathtools}	

\title{Motivation for phD-ansøgning}
\author{Kristian Knakkergaard Nielsen}

\begin{document}
\maketitle
Jeg har haft en interesse for naturen, så længe jeg kan huske. I mine første skoleår havde jeg en klar idé om, at jeg skulle være marinebiolog. Især blåhvaler var spændende med deres enorme størrelse og lidt hemmelige eksistens. Sidenhen, da vi fik Fysik og Kemi i de højere klasser, begyndte jeg at interesse mig for strøm og magneter. Især det sidstnævnte finder jeg den dag i dag utroligt fascinerende. Da jeg begyndte i gymnasiet var jeg således ret sikker på, at jeg ville læse fysik på universitetet. Dette vel at mærke uden egentlig at vide hvad fysik er. Det blev jeg (en smule) klogere på i løbet af gymnasiet, især studieretningsprojektet om Heisenbergs usikkerhedsrelationer i kvantemekanik gav mig blod på tanden. Siden da har jeg været dybt fascineret af kvantemekanik og de spøjse fænomener den forudser. 
\\

Nu ansøger jeg så om en phD her ved Aarhus Universitet. Et sted, hvor jeg har fået mine tanker om naturen sat i system og har lært, hvordan vi kan analysere og (forhåbentlig) forstå den. Det er derfor med største fornøjelse og iver, at jeg indgiver min phD-ansøgning. At få lov til at forske i den forunderlige verden af Bose-Einstein kondensater vil for mig være kulminationen på min nysgerrighed på naturen. En nysgerrighed der kun er blevet større i takt med, at jeg har bevæget mig fra et naivt barn, der drømmer om blåhvaler, til en analytisk fysiker, der vil manipulere med kondensater. 
\vspace{1cm}

Forhåbentlige hilsner,
\\

BSc Kristian Knakkergaard Nielsen

\end{document}